\documentclass[]{article}
\usepackage{amsmath,amsthm}
\usepackage{amsfonts}
\usepackage{color}

%opening
\title{Complex Exponential Function}
\author{Vincent La}

\begin{document}

\maketitle

\section{Complex Exponential Function}
The exponential function, when extended to the complex numbers, is defined as
\[e^z = e^{x}\cos(y) + ie^{x}\sin(y) \]

\subsection{$e^z$ is entire}
\begin{itemize}
	\item \textbf{Proposition 3.2} If $f_x$ and $f_y$ exist, are continuous in an neighborhood of $z$, and satisfying the Cauchy-Riemann equations, then $f$ is differentiable there.
	\item \textbf{Definition:} $f$ is entire if it is differentiable everywhere. In terms of the above proposition, that means its partial derivatives are continuous and satisfy the Cauchy-Riemann equations everywhere.
\end{itemize}

\begin{proof}
	To prove that $e^z$ is entire, we need to first show that its partial derivatives satisfy the Cauchy-Riemann Equations.
	
	\[\begin{aligned}
		u_y &= -e^x\sin(y) \\
		-v_x &= -e^x\sin(y) \\ 
		\\
		u_x &= e^x\cos(y) \\
		v_y &= e^x\cos(y) \\
	\end{aligned}\]
	
	Furthermore, we know that these partial derivatives are products of continuous functions, so they themselves are continuous. Because $e^z$ has continuous partial derivatives everywhere satisfying the Cauchy-Riemann Equations, it is differentiable everywhere. Therefore, it is an entire function.
\end{proof}

\section{Complex Sine and Cosine}
Let $z \in \mathbb{C}$
\[\cos(z) = \frac{e^{iz} + e^{-iz}}{2}\]
\[\sin(z) = \frac{e^{iz} - e^{-iz}}{2i}\]
	
\subsection{Derivatives}
\[\frac{d}{dz} \cos(z) = -\sin(z) \]

\begin{proof}
	\[\begin{aligned}
		\frac{d}{dz} \cos(z)
		&= \frac{d}{dz} \frac{e^{iz} + e^{-iz}}{2} \\
		&= \frac{ie^{iz} - ie^{-iz}}{2} \\
		&= \frac{i(e^{iz} - e^{-iz})}{2} \cdot \color{blue}\frac{i}{i} \\
		&= \frac{i^2(e^{iz} - e^{-iz})}{2i} \\
		&= -(\frac{e^{iz} - e^{-iz}}{2i}) = -\sin(z)
	\end{aligned}\]
\end{proof}

\end{document}