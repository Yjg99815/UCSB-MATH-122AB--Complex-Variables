\documentclass[]{article}
\usepackage[margin=1.25in]{geometry}
\usepackage{amsmath,amsthm}
\usepackage{graphicx,tikz}

%opening
\title{Poles, Removable and Essential Singular Points}
\author{Vincent La}

\begin{document}

\maketitle

\section{Definitions}
\paragraph{Principal Part} The terms in the Laurent series expansion of a function that contain negative exponents

\paragraph{Pole} A singular point that contains only a finite number of terms with negative exponents

\paragraph{Essential Singular Point} A singular point with an infinite number of terms with negative exponents

\paragraph{Removable Singular Point} A singular point whose Laurent expansion contains only terms with positive exponents. All removable singular points have a residue of 0.

\section{Exercises}
\textbf{Source:} Complex Variables and Applications, Section 56

\begin{enumerate}
	\item[2.] Show that the singular point of each of the following functions is a pole. Determine the order $m$ of that pole and the corresponding residue $B$.
	
	\begin{enumerate}
		\item
		\item $\frac{1 - \exp{2z}}{z^4}$
		
	\paragraph{Solution}
	Recall the MacLaurin series
	\[e^z = \sum^{\infty}_{n=0} \frac{z^n}{n!} (|z| < 1) \]
	
	Then, write
	\[\begin{aligned}
		\frac{1 - \exp{2z}}{z^4}
		&= \frac{1}{z^4} - \frac{\exp{2z}}{z^4} \\
		&= \frac{1}{z^4} - \sum^{\infty}_{n=0} \frac{\frac{(2z)^n}{n!}}{z^4} \\
		&= \frac{1}{z^4} - \sum^{\infty}_{n=0} \frac{(2z)^n}{n! \cdot z^4} \\
		&= \frac{1}{z^4} - \sum^{\infty}_{n=0} \frac{2^n z^n}{n! \cdot z^4} \\
		&= \frac{1}{z^4} - \sum^{\infty}_{n=0} \frac{2^n z^{ n - 4 }}{n!} \\
	\end{aligned}\]
	
	Expanding the series, we get
	\[\begin{aligned}
		\frac{1 - \exp{2z}}{z^4}
		&= \frac{1}{z^4} - \sum^{\infty}_{n=0} \frac{2^n z^{ n - 4 }}{n!} \\
		&= \frac{1}{z^4} - [ \frac{2^0z^{-4}}{0!} + \frac{{2^1}z^{-3}}{1!}
			+ \frac{2^2z^{-2}}{2!} + \mathbf{ \frac{2^3 z^{-1}}{3!} } + ... ] \\	
	\end{aligned}\]
	
	Therefore, the $b_1$ term of the Laurent expansion (and therefore our residue) is
	$-\frac{2^3}{3!} = -\frac{8}{6} = -\frac{4}{3}$.
		
	\paragraph{Answers}
	(Provided by the book)
	\begin{enumerate}
		\item $m = 1, B = \frac{-1}{2}$,
		\item $m = 3, B = \frac{-4}{3}$
		\item $m = 2, B = 2e^2$
	\end{enumerate}
		
	\end{enumerate}
\end{enumerate}

\end{document}