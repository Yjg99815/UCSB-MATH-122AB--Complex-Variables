\documentclass[11pt]{article}
\usepackage[margin=1.25in]{geometry}

\usepackage{graphicx,tikz}
\usepackage{amsmath,amsthm}
\usepackage{amsfonts}
\usepackage{amssymb}
\usepackage{boondox-cal}
\title{ }

\newtheorem*{thm}{Theorem}

\begin{document}
	%\maketitle
	%\date
	\begin{center}	% centers
		\Large{Homework 3}	% Large makes the font larger, put title inside { }
	\end{center}
	\begin{center}
		Vincent La \\
		Math 122A \\
		July 15, 2017
	\end{center}

\section{Monday}
\begin{enumerate}
	\item[Extra 5.] Write down in terms of $M$-neighborhoods of $\infty$ what it means that $\lim_{z \rightarrow \infty} f(z) = \infty$.
	
	\paragraph{Answer} When we project the complex numbers on a sphere, we can see that as the magnitude of the number increases more and more they seem to converge to the north pole of the sphere. 
	
	\item[Extra 6.] Prove that $\lim_{z \rightarrow 0}{\frac{1}{z}} = \infty$ and that $\lim_{z \rightarrow 0} \frac{1}{z} = \infty$ and that $\lim_{z \rightarrow \infty} \frac{1}{z} = 0$. 
	
	\begin{enumerate}
		\item Prove that $\lim_{z \rightarrow 0}{\frac{1}{z}} = \infty$
		
		\begin{proof}
			Let $M > 0$, $z \in \mathbb{C}$ be arbitrary. Then pick $\delta = \frac{1}{M}$. 
			
			As a result, whenever $|z - 0| = |z| < \delta$, we get that
			\[|z| < \frac{1}{M}\]
			
			Now, multiply both sides by $M$, and we get
			\[|z|M < 1\]
			
			Divide both sides by $|z|$, and we get
			\[M < \frac{1}{|z|} = f(z)\]
			
			as we set out to prove.		
		\end{proof}
		
		\item Prove that $\lim_{z \rightarrow \infty} \frac{1}{z} = 0$
		
		\begin{proof}
			 We want to show that for any $\epsilon > 0$, there is a $\delta > 0$ so that if $|z| > \delta$ then $|f(z) - 0| < \epsilon$. Let $\delta = \frac{1}{\epsilon} >0$.
			 
			 \bigskip
			 
			 If $|z| > \delta$, then $|z| > \frac{1}{\epsilon}$. If we take the reciprocal of both sides, we get
			 \[f(z) = \frac{1}{|z|} < \epsilon \]
			 
			 as we set out to prove.			 
		\end{proof}
	\end{enumerate}
	
	\item[Extra 7.] A complex values function $f: D \rightarrow \mathbb{C}$ is called bounded if there exists $M > 0$ such that $|f(z)| < M$ for all $z \in D$. Prove that $\{z_n\} \subseteq D$ such that $f(z_n) \rightarrow \infty$.
\end{enumerate}

\section{Tuesday}
\begin{enumerate}
	\item[2.3] By comparing coefficients or by use of the Cauchy-Riemann equations, determine which of the following polynomials are analytic.
	
	\begin{enumerate}
		\item $P(x + iy) = x^3 - 3xy^2 - x + i(3x^2y - y^3 - y)$
		
		\paragraph{Analytic}
		
		Here, 
		\[u(x, y) = x^3 - 3xy^2 - x\] and
		\[v(x, y) = 3x^2y - y^3 - y\]. Therefore,
		
		\[\begin{aligned}
		u_x &= 3x^2 - 3y^2 - 1 \\
		v_y &= 3x^2 - 3y^2 - 1 \\		
		u_y &= -6xy \\
		-v_x &= -6xy
		\end{aligned}		
		\]
		
		Because $u_x = v_y$ and $u_y = -v-x$, $P(x + iy)$ is analytic.
		
		\item $P(x + iy) = x^2 + iy^2$
		
		\paragraph{Not Analytic}
		
		Here, $u(x, y) = x^2$ and $v(x, y) = y^2$. Therefore, 
		
		\[\begin{aligned}
		u_x &= 2x \\
		v_y &= 2y \\ 
		\end{aligned}\]
		
		implying that $u_x \neq v_y$ or that $P(x + iy)$ is not analytic.
						
		\item $P(x + iy) = 2xy + i(y^2 - x^2)$
		
		\paragraph{Analytic}
		
		Here, $u(x, y) = 2xy$ and $v(x, y) = y^2 - x^2$. Therefore,
		
		\[\begin{aligned}
		u_x &= 2y \\
		v_y &= 2y \\
		u_y &= 2x \\
		-v_x = -(-2x) = 2x \\
		\end{aligned}\]
		
		Because $u_x = v_y$ and $u_y = -v_x$, $P(x + iy)$ is analytic.
	\end{enumerate}
	
	\item[2.4] Show that no nonconstant analytic polynomial can take imaginary values only.
	
	\begin{proof}
		Let us prove the contrapositive, that if $f$ takes only imaginary values then $f$ is either constant or not analytic.
		
		\bigskip
		
		If $f$ only takes imaginary values then it has no real component, i.e. $u(x, y) = 0$. Thus, $u_x = 0$ and $u_y = 0$. If $f$ is analytic, then necessarily $v_x = v_y = 0$. Because, only a constant function has these derivatives $f$ is constant.
		
		\bigskip
		
		Now, if we relax the requirement that $f$ satisfy the Cauchy-Riemann equations, then it is free to be non-constant. However, it will not be analytic, as we set out to prove.
	\end{proof}
\end{enumerate}

\section{Wednesday and Thursday}
\begin{enumerate}
	\item[3.2] Show $f(z) = x^2 + iy^2$ is differentiable at all points on the line $x = y$. 
	\begin{proof}
		Let $z \in \mathbb{C}$ be arbitrary as long as $x = y$. Then, we want to show that $f$ is differentiable, or in other words $\lim_{h \rightarrow 0} \frac{f(z+h) - f(z)}{h}$ exists.
		
		Let us write $z = x_1 + iy_1$ and $h \in \mathbb{C}$ as $h = x_2 + iy_2$. Then, our limit becomes
		\[\begin{aligned}
		& \lim_{x_2 + iy_2 \rightarrow 0} \frac{
			f(x_1 + iy_1 + x_2 + iy_2) -
			f(x_1 + iy_1)}{
			x_2 + iy_2} \\
		& \lim_{x_2 + iy_2 \rightarrow 0} \frac{
			f[x_1 + x_2 + i(y_1 + y_2)] -
			f(x_1 + iy_1)}{
			x_2 + iy_2} \\
		& \lim_{x_2 + ix_2 \rightarrow 0} \frac{
			f[x_1 + x_2 + i(x_1 + x_2)] -
			f(x_1 + ix_1)}{
			x_2 + ix_2} \\
		& \lim_{x_2 + ix_2 \rightarrow 0} \frac{
			(x_1 + x_2)^2 + i(x_1 + x_2)^2 -
			(x_1^2 + ix_1^2)}{
			x_2 + ix_2} \\
		& \lim_{x_2 + ix_2 \rightarrow 0} \frac{
			x_1^2 + 2x_1x_2 + x_2^2 + ix_1^2 + 2ix_1x_2 + ix_2^2 -
			(x_1^2 + ix_1^2)}{
			x_2 + ix_2} \\
		& \lim_{x_2 + ix_2 \rightarrow 0} \frac{
			2x_1x_2 + x_2^2 + 2ix_1x_2 + ix_2^2}{
			x_2 + ix_2} \\
		& \lim_{x_2 + ix_2 \rightarrow 0} \frac{
			2x_1x_2(1 + i) + x_2^2(1 + i)}{
			x_2(1 + i)} \\
		& \lim_{x_2 + ix_2 \rightarrow 0} \frac{
			2x_1x_2 + x_2^2}{
			x_2} \\
		& \lim_{x_2 + ix_2 \rightarrow 0} 2x_1 + x_2 = 2x_1 + x_2 \\
		\end{aligned}\]
		Therefore, our limit exists so $f$ is differentiable.
		
		\paragraph{Nowhere Analytic} Show that $f$ is nowhere analytic.
		
		Using the Cauchy-Riemann Equations, we see that 
		\[u_x = 2y, v_y = 2y, u_y =0, -v_x = 0\]
		
		Therefore, $f$ is analytic only on the line $x = y$. However, for $f$ to be analytic for any $z$, it must be analytic in a neighborhood of $z$. However, for any $z$ where $f$ is analytic, its neighborhood contains an infinite number of points where $x \neq y$. Therefore, $f$ is nowhere analytic.
	\end{proof}
	
	\item[3.5] Suppose $f$ is analytic in a region and $f' \equiv 0$ there. Show that $f$ is constant. 
	
	\begin{proof}
		Let $f$ be any function analytic in a region where $f' \ equiv 0$ in that region. Because it is analytic, it satisfies the Cauchy-Riemann Equations, implying $u_x = v_y$ and $u_y = -v_x$. Because $f' \equiv 0$, it must mean that its partial derivatives either cancel each other out or are equal to 0. However, because $f$ is analytic, the conditions imposed by the Cauchy-Riemann Equations imply $u_x = v_y = 0$ and $u_y = -v_x =0$. Only a constant function has these derivatives, so $f$ is constant.
	\end{proof}
	
	\item[3.9] Show that there are no analytic functions $f = u + iv$ with $u(x, y) = x^2 + y^2$. 
	
	\begin{proof}
		Let $f$ be an arbitrary function with $u(x, y) = x^2 + y^2$, and assume for a contradiction that it is analytic.
		
		\bigskip
		
		Because $f$ is analytic, it satisfies the Cauchy-Riemann Equations. Because $u_x = 2x$ and $u_y = 2y$, this implies $v_y = 2x$ and $-v_x = 2y$. Now, notice that $v_y = 2x$ implies 
		\[\int v_y dy = \int 2x dy\]
		\[v = 2xy + c(y) \]
		On the other hand, $v_x = -2y$ implies
		\[\int v_x dx = \int -2y dx \]
		\[v = -2xy + c(x)\]
		
		However, $v = 2xy + c(y)$ and $v = -2xy + c(x)$ cannot be true at the same time. Therefore, there are no analytic functions $f = u + iv$ with $u(x, y) = x^2 + y^2$.		
	\end{proof}
	
	\item[3.10] Suppose $f$ is an entire function of the form
		\[f(x, y) = u(x) + iv(y)\]
		
	Show that $f$ is a linear polynomial.
	
	\begin{proof}
		Let $f$ be any entire function of the form $u(x) + iv(y)$. If $f$ is entire, it is analytic anywhere, so it constantly satisfies the Cauchy Riemann equations. Because $u_y = -v_x = 0$ for all $x, y$ this imposes the constraint that $u_x = v_y$ for all $x, y$.
		
		\bigskip
		
		Suppose for a contradiction that $u(x), v(y)$ are not linear polynomials, i.e. they are polynomials with degree $> 1$. However, if we take partial derivatives $u_x$ and $v_y$, then they will necessarily have some term involving $x$ and $y$ respectively. Now, take for example--$x = 0, y = 1$, then their partial derivatives evaluate to $u_x= 0, v_y= 1$. This contradicts the assumption $f$ is entire, so this cannot be the case.
		
		\bigskip
		
		Now, we are left with the possibility that $u(x), v(y)$ are both linear polynomials. Write $u(x) = a + bx$ and $v(y) = c + dy$. Taking partial derivatives, we get $u_x = b$ and $v_y = d$. Thus, the Cauchy-Riemann Equations are satisfied if $x$ and $y$ have the same coefficients.
		
		\bigskip
		
		In conclusion, we have shown that any entire function of the form $u(x) + iv(y)$ has $u(x)$ and $v(y)$ as linear polynomials sharing the same coefficient on the first degree term.
	\end{proof}
\end{enumerate}
\end{document}