\documentclass[11pt]{article}
\usepackage[margin=1.25in]{geometry}

\usepackage{graphicx,tikz}
\usepackage{amsmath,amsthm}
\usepackage{amsfonts}
\usepackage{amssymb}
\usepackage{boondox-cal}
\title{ }

\newtheorem*{thm}{Theorem}

\begin{document}
	%\maketitle
	%\date
	\begin{center}	% centers
		\Large{Homework 5}	% Large makes the font larger, put title inside { }
	\end{center}
	\begin{center}
		Vincent La \\
		Math 122B \\
		August 20, 2017
	\end{center}
	
\section{Theorems}
\paragraph{Theorem 1} If f is analytic everywhere in the finite plane
except for a finite number of points interior to a positively oriented
simple closed contour $C$, then
\[\int_C f dz = 2\pi i Res_{z=0} [\frac{1}{z^2} \cdot f(\frac{1}{z})] \]
	
\section{Homework}
\begin{enumerate}
	\item Use Cauchy's Residue Theorem to evaluate the integral around the circle
	$|z| = 3$ with an orientation of
	\begin{enumerate}
		\item $f(z) = z^2\exp{\frac{1}{z}}$
		\paragraph{Solution}
		This function is analytic everywhere except $z = 0$, where there is a singularity. It is isolated because it is analytic in an neighborhood around it, and it is the only singular point of the function.
		
		\bigskip
		
		Now, let us find the Laurent Series for this function. Recall the MacLaurin series
		\[\begin{aligned}
		\exp{z} &= \sum^{\infty}_{n=0} \frac{z^n}{n!} & (|z| < \infty) \\
		\exp{\frac{1}{z}}
		&= \sum^{\infty}_{n=0} \frac{(\frac{1}{z}^n)}{n!} \\
		&= \sum^{\infty}_{n=0} \frac{1}{z}\frac{1}{n!} & (0 < |z| < \infty) \\
		\end{aligned}\]
		
		Therefore,
		\[\begin{aligned}
		z^2\exp{\frac{1}{z}}
		&= z^2 \cdot [
			\frac{1}{1!} \cdot \frac{1}{z^0} + 
			\frac{1}{1!} \cdot \frac{1}{z} + 
			\frac{1}{2!} \cdot \frac{1}{z^2} + 
			\frac{1}{3!} \cdot \frac{1}{z^3} +
			\frac{1}{4!} \cdot \frac{1}{z^4} + ...] \\
		&= z^2 \cdot [1 + z + \frac{1}{2!z^2} +
			\frac{1}{3!z^3} + \frac{1}{4!z^4} + ...] \\
		&= z^3 + z^2 + \frac{1}{2!} + \frac{1}{3!z} +
			\frac{1}{4!z^2} + ...
		\end{aligned}\]
		
		So, our residue $b_1 = \frac{1}{3!} = \frac{1}{6}$. Because
		$\int_C f(z) dz = 2\pi i b_1$, this implies
		\[\int_C f(z) dz = \frac{1}{3} \pi i \]
		
		\item $f(z) = \frac{z+1}{z^2 - 2z}$
		
		\paragraph{Solution}
		This function has two singularities, $z = 0$ and $z = 2$.
		
		\paragraph{$\mathbf{z = 0}$}
		First, recall that the Laurent Series coefficients are given by 
		\[\begin{aligned}
		b_n &= \frac{1}{2\pi i} \int_C \frac{f(z)}{(z - z_0)^{-n + 1}} dz \\
		b_1 &= \frac{1}{2\pi i} \int_C f(z) dz \\
		\end{aligned}\]
		
		Thus, our residue at $z = 0$ is
		\[\begin{aligned}
		\frac{1}{2\pi i} \cdot \int_C \frac{z + 1}{z^2 - 2z} dz &=
		\frac{1}{2\pi i} \cdot [\int_C \frac{z}{z^2 - 2z} dz +
			\int_C \frac{1}{z^2 - 2z} dz] \\
		&= \frac{1}{2\pi i} \cdot [\int_C \frac{1}{z - 2} dz +
		\int_C \frac{1}{z^2 - 2z} dz] \\
		&= \frac{1}{2\pi i} \cdot [\int_C \frac{1}{z - 2} dz +
		\int_C \frac{-1}{2} \cdot \frac{1}{z} dz +
		\int_C \frac{1}{2} \cdot \frac{1}{z - 2} dz ] & \text{Partial fractions} \\
		&= \frac{1}{2\pi i} \cdot
		[\frac{3}{2} \int_C \frac{1}{z - 2} dz +
         \frac{-1}{2} \int_C \frac{1}{z} dz ]
		\end{aligned}\]
		
		For the first integral, according to Cauchy's Integral Formula, $f(z) = 1$ and $z_0 = 0$. Because $f(z) = 1$ is obviously analytic, and $0$ is interior to the circle $|z| < 3$, then 
		\[\int_C \frac{1}{z - 2} dz = 2\pi i f(2) = 2\pi i\]
		
		For the second integral above, we can also apply Cauchy's Integral Formula and get
		\[\int_C \frac{1}{z - 0} dz = 2\pi i f(0) = 2\pi i \]	
		
		In conclusion,
		\[\begin{aligned}
		\frac{1}{2\pi i} \cdot \int_C \frac{z + 1}{z^2 - 2z} dz
		&= \frac{1}{2\pi i} \cdot [\frac{3}{2} \cdot 2\pi i + \frac{-1}{2} \cdot 2\pi i] \\
		&= 1 = Res_{z = 0}
		\end{aligned}\]

		\paragraph{$\mathbf{z = 2}$}
		Notice that the same formulas above also apply for $z = 2$. Therefore, 
		\[Res_{z = 2} = 1\]
		
		\paragraph{Conclusion}
		In conclusion, $\int_C \frac{z+1}{z^2 - 2z} = 2\pi i(Res_1 + Res_2) = 2\pi i \cdot 2 = 4\pi i$.
	\end{enumerate}
	
	\item Use the theorem involving one residue to evaluate the following integrals over the circle $|z| = 2$ with positive orientation.
	
	\begin{enumerate}
		\item $f(z) = \frac{1}{1 + z^2}$
		\paragraph{Solution} Because $\frac{1}{1 + z^2}$ is analytic everywhere exccept for $z = \pm i$, we can use Theorem 1 to evaluate this integral.
		
		First, let us find the residue at $z = 0$. Recall the MacLaurin series,
		\[\begin{aligned}
			\frac{1}{1 + z}
			&= \sum^{\infty}_{n=0} (-1)^n \cdot z^n & (|z| < 1) \\
			\frac{1}{1 + z^2}
			&= \sum^{\infty}_{n=0} (-1)^n \cdot z^{2n} & \text{Plug $z = z^2$ into above } (|z| < 1) \\
		\end{aligned}\]
		
		If we expand the series above, we get no $\frac{b_1}{z - z_0}$ terms, so the residue is 0. Therefore, $\int_C f(z) = 0$.
		
		\item $f(z) = \frac{1}{z}$
		
		\paragraph{Solution} Because $\frac{1}{z}$ is analytic everywhere except for $z = 0$, we can use Theorem 1 to evaluate this integral.
		
		\[\begin{aligned}
		\int_C f(z) d(z) &=
		2\pi i \cdot Res_{z=0} [\frac{1}{z^2} \cdot f(\frac{1}{z})] \\
		&= 2\pi i \cdot Res_{z=0} [\frac{1}{z^2} \cdot z] \\
		&= 2\pi i \cdot Res_{z=0} [\frac{1}{z}]
		\end{aligned}\]

		Unforunately, because $\frac{1}{z}$ is not analytic at $z = 0$ we cannot use the MacLaurin series expansion. Again, use the formula 
		$b_1 = \frac{1}{2\pi i} \int_C \frac{1}{z} dz$. Then, consider the parameterization of $|z| < 2$ given by $|z| = 2e^{i\theta}$ for $0 < \theta < 2\pi$.
		
		\[\begin{aligned}
		b_1 = \frac{1}{2\pi i} \oint z^{-1} dz
		&= \frac{1}{2\pi i} \int_0^{2\pi} (2e^{i\theta})^{-1} \cdot (-2\sin{\theta} + 2i\cos{\theta}) d\theta \\
		&= \frac{1}{2\pi i} \int_0^{2\pi} 
			\frac{-2\sin{\theta} + 2i\cos{\theta}}{2e^{i\theta}} d\theta \\
		&= \frac{1}{2\pi i} \int_0^{2\pi} 
		\frac{-\sin{\theta} + i\cos{\theta}}{e^{i\theta}} d\theta \\
		&= \frac{1}{2\pi i} \int_0^{2\pi} 
		\frac{ie^{i\theta}}{e^{i\theta}} d\theta \\
		&= \frac{1}{2\pi i} \int_0^{2\pi} i d\theta \\
		&= \frac{1}{2\pi i} \cdot [2\pi i] \\
		&= 1
		\end{aligned}\]
		
		In conclusion, $\int_C f(z) d(z) = 2\pi i$.
		
	\end{enumerate}
\end{enumerate}
\end{document}